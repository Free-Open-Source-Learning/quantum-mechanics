\chapter{Equazione di Schrödinger nel caso 3d e potenziali centrali}

\section{Equazione di Schrödinger nel caso 3d}

Trascurando eventuali dipendenze temporali, possiamo scrivere

\begin{equation}
\hat H \psi(\underline r)= \left(
\frac{\hat p^2}{2m} + V(\underline r)
\right)\psi(\underline r)= E\psi(\underline r)
\end{equation}

Essendo ora in 3d avremo che

\begin{align}
\hat{\underline p} = -i\hbar \left(
\frac{\partial}{\partial x}, \frac{\partial}{\partial y}, \frac{\partial}{\partial z}
\right) \rightarrow 
\hat{p^2} = -\hbar^2 \left(
\frac{\partial^2}{\partial x ^2} + \frac{\partial^2}{\partial y^2} + \frac{\partial^2}{\partial z^2}
\right) = - \hbar^2 \nabla^2
\end{align}

da cui otteniamo

\begin{align}
 \frac{\hbar^2}{2m} \nabla^2\psi(x,y,z) - [V(x,y,z) - E]\psi(x,y,z)=0
\end{align}

Ma come si studia questo problema? In alcuni casi è possibile studiare il problema nel caso 1d lungo gli assi per poi sommare i contributi, e in questo caso si parla di \textbf{metodo della separazione di variabili}:

\begin{equation}
H(x,y,z)= H(x) + H(y) + H(z)
\end{equation}


Esempi di sistemi che rientrano in questa categoria sono
\begin{enumerate}
	\item la buca di potenziale a parallelogramma (la "scatola")
	\item l'oscillatore armonico 3d (isoptropo e anisotropo)
\end{enumerate}

Qualora non sia possibile il problema si complica.

\newpage

\section{Potenziali centrali}

Si definiscono tali i potenziali tali che

\begin{equation}
V(r,\theta,\phi) \neq V(r)
\end{equation}

Per studiare questo genere di problemi ci conviene passare in coordinate sferiche. Scriviamo quindi:

\begin{align}
\nabla^2 {}&= \frac{\partial^2}{\partial x ^2} + \frac{\partial^2}{\partial y^2} + \frac{\partial^2}{\partial z^2}  \nonumber \\
&\downarrow \nonumber \\
\nabla^2 &=\frac{1}{r^2}\frac{\partial}{\partial r}\left(  r^2  \frac{\partial}{\partial r}  \right) + 
\frac{1}{r^2 \sin{\theta}}\, \frac{\partial}{\partial \theta}\left(  \sin{\theta} \; \frac{\partial}{\partial \theta}  \right) + 
\frac{1}{r^2 \sin^2{\theta}}\frac{\partial^2}{\partial \phi^2}
\end{align}

ricordando che

\begin{align}
L^2= - \hbar^2 \left[
\frac{1}{\sin{\theta}}\, \frac{\partial}{\partial \theta}\left(  \sin{\theta} \; \frac{\partial}{\partial \theta}  \right) +
\frac{1}{\sin^2{\theta}}\frac{\partial^2}{\partial \phi^2}
\right]
\end{align}

da cui

\begin{align}
\nabla^2 = \frac{1}{r^2}\frac{\partial}{\partial r}\left(  r^2  \frac{\partial}{\partial r}  \right) - \frac{1}{\hbar^2 r^2}L^2
\end{align}

Possiamo quindi riscrivere l'eq di Schrödinger come

\begin{equation}
\frac{\hbar^2}{2 m}
\left[
\frac{1}{r^2}\frac{\partial}{\partial r}\left(  r^2  \frac{\partial}{\partial r}  \right) - \frac{1}{\hbar^2 r^2}L^2
\right]\psi(r,\theta,\phi) - [V - E]\psi(r,\theta,\phi)=0
\end{equation}

Ricordiamo che il momento angolare non ha dipendenze legate alla distanza dalla sorgente del potenziale:

\begin{align}
\left\{
\begin{array}{ccc}
L_i \neq f(r) \\
L^2 \neq f(r)
\end{array}
\right. 
\rightarrow
\left\{
\begin{array}{ccc}
\left[L_i,V(r) \right]=0 \\
\left[L^2,V(r)\right]=0
\end{array}
\right. 
\end{align}

e inoltre sappiamo che

\begin{align}
\left\{
\begin{array}{ccc}
\left[L_i,H \right]=0 \\
\left[L^2,H\right]=0
\end{array}
\right.
\rightarrow
[H,V(r)]=0
\end{align}

Possiamo quindi scrivere le autofunzioni di H tramite quelle di $L^2$ e cercare soluzioni all'equazione di Schrödinger nella forma

\begin{equation}
\psi(r,\theta,\phi)= R_{El}(r)Y_l^m(\theta,\phi)
\end{equation}

Notiamo come la parte radiale dipenda da $E$ e da $l$ ma non da $m$. QUesto è dovuto dal fatto che nell'eq. di Schrödinger in $\nabla^2$ compaia $L^2$ ma non $L_z$. \newline
\newpage
Vediamo ora come si comporta l'hamiltoniana:


\begin{align}
{}&\left(-\frac{\hbar^2}{2 m}
\left[
\frac{1}{r^2}\frac{\partial}{\partial r}\left(  r^2  \frac{\partial}{\partial r}  \right) - \frac{1}{\hbar^2 r^2}L^2
\right]   + V \right) R_{El}(r)Y_l^m(\theta,\phi) = E R_{El}(r)Y_l^m(\theta,\phi) \nonumber \\
&\downarrow \nonumber \\
&\left(-\frac{\hbar^2}{2 m}
\left[
\frac{1}{r^2}\frac{\partial}{\partial r}\left(  r^2  \frac{\partial}{\partial r}  \right) - \frac{1}{\hbar^2 r^2}L^2
\right]   + V \right) R_{El}(r) = E R_{El}(r) \nonumber \\
&\downarrow \nonumber \\
& \left( -\frac{\hbar^2}{2 m} \frac{1}{r^2}\frac{\partial}{\partial r}\left(  r^2  \frac{\partial}{\partial r}  \right) + \left[ V(r) +  \frac{1}{2m r^2}L^2 \right] \right) R_{El}(r) = E R_{El}(r) \nonumber \\
&\downarrow \nonumber \\
&\left( -\frac{\hbar^2}{2 m} \frac{1}{r^2}\frac{\partial}{\partial r}\left(  r^2  \frac{\partial}{\partial r}  \right) + V_{\text{eff}}(r) \right) R_{El}(r) = E R_{El}(r)
\end{align}

Per risolvere questa equazione ci conviene definire:

\begin{equation}
u_{E l}(r)= r R_{E l}(r) \rightarrow R_{E l}(r)= \frac{u_{E l}(r)}{r}
\end{equation}

dalla quale avremmo

\begin{align}
\left\{
\begin{array}{cc}
\frac{d R_{E l}(r)}{dr} = \frac{1}{r}\frac{d u_{E l}(r)}{d r} - \frac{1}{r^2}u^2_{El}(r)  \qquad \qquad \quad \: \\
\frac{d^2 R_{E l}(r)}{dr^2} = \frac{1}{r^2}\frac{d^2 u_{E l}(r)}{dr^2} - \frac{2}{r^2}\frac{d u_{E l}(r)}{dr} + \frac{2}{r^3} u_{E l}(r)
\end{array}
\right.
\end{align}

da cui otteniamo infine

\begin{align}
\left\{
\begin{array}{cc}
-\frac{\hbar^2}{2 \mu} \frac{d^2 }{dr^2} u_{E l}(r) + V_{\text{eff}}(r)u_{E l}(r) = E u_{E l}(r) \\
V_{\text{eff}}= V(r) +  \frac{l(l+1)\hbar^2}{2m r^2} \qquad\qquad\qquad
\end{array}
\right.
\end{align}

Notiamo nel $V_{eff}$ come il momento angolare dia un contributo repulsivo.
\newpage
\section{Problemi a due corpi}
I problemi a due corpi possono essere studiati in due modi:

\subsection{Sistema di riferimento fisso esterno} 
	\begin{align}
	{}&H= -\frac{\hbar^2}{2 m_1}\nabla_1^2 -\frac{\hbar^2}{2 m_2}\nabla_2^2 + V(\underline{r}) \\ 
	&\underline{r}= \underline{r}_1 - \underline{r}_2
	\end{align}

\subsection{Nel sistema del centro di massa}
	
	
	\begin{align}
	{}&H= -\frac{\hbar^2}{2 M}\nabla_R^2 -\frac{\hbar^2}{2 \mu}\nabla_{\underline r}^2 +  V(\underline{r}) \\
	&\mu = \frac{m_1 m_2}{m_1 + m_2} \quad ; \quad R= \frac{m_1 r_1 + m_2 r_2}{m_1 + m_2}
	\end{align}

In questo caso si divide lo studio del problema in due parti:

\begin{enumerate}
	\item moto  \textbf{del centro di massa}, descritto dal primo termine
	\item moto \textbf{relativo}, descritto dal secondo termine 
\end{enumerate}

e conseguentemente la funzione d'onda può essere scomposta nel seguente modo

\begin{equation}
\psi(r,R)= f(r)g(R)
\end{equation}

La soluzione del \textbf{centro di massa} si ricava facilmente, dipendendo da R solo il primo dei tre termini di H:

\begin{align}
g(R)= \frac{1}{\sqrt{2\pi}} e^{i\frac{kR}{\hbar}} \quad ; \quad k = \sqrt{2mE}
\end{align}

Per quanto riguarda invece il \textbf{moto relativo} invece, passando in coordinate polari, il problema assume la forma

\begin{align}
H \psi(r,\theta,\phi)= \left[
-\frac{\hbar^2}{2 \mu}\frac{\partial^2}{\partial r^2} + V(r) +  \frac{\hbar^2 l(l+1)}{2 \mu r^2} \right] u_{E l}(r) = E \, u_{E l}(r)
\end{align}

Continuiamo lo studio del problema utilizzando il caso particolare dell'atomo d'idrogeno nel prossimo paragrafo.

\newpage

\section{L'atomo d'Idrogeno}

L'atomo d'Idrogeno è un sistema a due corpi composto da un nucleo a singolo protone e un elettrone orbitante. In questo caso il potenziale in gioco è di tipo coulombiano:

\begin{align}
V(r)= \frac{Z q^2}{4\pi \epsilon_0 r}
\end{align}

e quindi, ricordando che per l'idrogeno Z=1, l'Hamiltoniana del centro di massa assume la seguente forma:

\begin{align}
{}&H \psi(r,\theta,\phi)= \left[
-\frac{\hbar^2}{2 \mu}\frac{\partial^2}{\partial r^2} + \frac{q^2}{4\pi \epsilon_0 r} +  \frac{\hbar^2 l(l+1)}{2 \mu r^2} \right] u_{E l}(r) = E \, u_{E l}(r) \\
&\downarrow \nonumber \\ 
&\frac{\hbar^2}{2 \mu}\frac{\partial^2}{\partial r^2}u_{E l}(r)  + [E -V_{\text{eff}}] u_{E l}(r) = 0 \quad ; \quad
V_{\text{eff}}= \frac{q^2}{4\pi \epsilon_0 r} +  \frac{\hbar^2 l(l+1)}{2 \mu r^2} \end{align}

Abbiamo quindi un'equazione differeziale al 2o ordine a coefficienti non costanti. Dobbiamo quindi spezzare lo studio in due parti:

\begin{enumerate}
	\item $r \rightarrow 0$
	\item $r \rightarrow \infty$
\end{enumerate}


Nel primo caso si ha che $\frac{1}{r} \rightarrow \infty$. Considerando che quindi $\frac{1}{r^2} > \frac{1}{r}$ abbiamo che nel nel $V_{\text{eff}}$ il termine dominante è quello centrifugo repulsivo, e possiamo approssimare il problema a 

\begin{align}
\frac{\partial^2}{\partial r^2} u_{E l}(r) - \frac{\hbar^2 l(l+1)}{r^2} u_{E l}(r)=0
\end{align}

In questa approssimazione rientra nella famiglia delle \textbf{equazioni differenziali di Eulero} del tipo

\begin{equation}
x^2 y'', xy' +y=0
\end{equation}

con soluzioni nella forma 

\begin{equation}
y=x^m
\end{equation}

scriviamo quindi

\begin{align}
u_{E l}(r)= r^m \: ;\: \frac{\partial }{\partial r}u_{E l}(r) = m \, r^{m-1} \: ;\: \frac{\partial^2}{\partial r^2}u_{E l}(r) = m(m-1) \, r^{m-2}
\end{align}

Che inserite nella nostra equazione ci danno

\begin{align}
{}&m(m-1) \, r^{m-2} - \frac{\hbar^2 l(l+1)}{r^2}r^m= m(m-1) \, r^{m-2} -  \hbar^2 l(l+1)r^{m-2}=0 \nonumber\\
&\downarrow \nonumber \\
&[m(m-1) - l(l+1)]r^{m-2}=0
\end{align}


\newpage
cerchiamo quindi le soluzioni per $m^2 - m - l(l+1)=0$ che saranno della forma

\begin{align}
{}&m = \frac{1}{2}[1 \pm \sqrt{1 + 4l^2 + 4l}] = \frac{1}{2} [1 \pm \sqrt{(1 + 2l)^2}] = \frac{1}{2} [1 \pm (1 + 2l)]\\
&m_+ = l+1 \quad ; \quad m_-= -l
\end{align}

Questo significa che abbiamo due possibili soluzioni per la nostra equazione:

\begin{align}
u_{E l}(r) = \left\{
\begin{array}{cc}
r^{l+1}\\r^{-l}
\end{array}
\right.
\end{align}


Dobbiamo però tenere da conto che la nostra soluzione deve essere $\in L_2$, e deve perciò essere limitata. Questo esclude la seconda soluzione, che per $r=0$ esplode. Quindi in definitiva avremo che 
\begin{equation}
\lim_{r \rightarrow 0} u_{E l}(r)  \sim r^{l+1}
\end{equation}

Invece per $r \rightarrow +\infty$ il potenziale si annulla in toto, e scriviamo quindi:
\begin{align}
\frac{\partial^2}{\partial r^2} u_{E l}(r) + \frac{2 \mu E}{\hbar^2} u_{E l}(r) =0
\end{align}
Questa è una semplice equazione differenziale al 2o ordine a coefficienti costanti, la cui soluzione è della forma:
\begin{align}
u_{E l}(r) = Ae^{+kr} + Be^{-kr} \quad ; \quad k= \frac{\sqrt{-2 \mu E}}{\hbar}
\end{align}
per gli stessi motivi del paragrafo precedente escludiamo il primo termine, e quindi scriviamo che 
\begin{equation}
\lim_{r \rightarrow +\infty} u_{E l}(r)  \sim e^{-kr} \quad ; \quad k= \frac{\sqrt{-2 \mu E}}{\hbar}
\end{equation}
Torniamo ora al caso generale, e facendo le seguenti sostituzioni
\begin{align}
{}&k^2 = \frac{2 \mu E}{\hbar} \\
&\lambda = \frac{\mu q^2}{2 \pi \epsilon_0 \hbar^2}\frac{1}{k} \\
&\rho = kr \rightarrow \partial^2 \rho = k^2 \partial^2r
\end{align}
riscriviamo la nostra equazione:
\begin{align}
\frac{\partial^2}{\partial \rho^2} u_{E l}(\rho) + \left[  1 - \frac{\lambda}{\rho} - \frac{l(l+1)}{\rho^2} \right]u_{E l}(\rho)=0
\end{align}
Considerando quanto visto finora nei discorsi a limite, possiamo scrivere che
\begin{align}
 u_{E l}(\rho) = \rho^{l+1} e^{-\rho}V(\rho)
\end{align}
da cui
\begin{align}
\frac{\partial}{\partial \rho} u_{E l}(\rho) {}&= (l+1) \rho^l e^{-\rho}V(\rho) + \rho^{l+1}e^{-\rho}\frac{\partial}{\partial \rho} V(\rho) - \rho^{l+2} e^{-\rho}V(\rho)= \nonumber \\
&= \left[ (l+1-\rho)V(\rho) + \frac{\partial}{\partial \rho} V(\rho) \right] \rho^l e^{-\rho}
\end{align}
da cui segue
\begin{align}
\frac{\partial^2}{\partial \rho^2} u_{E l}(\rho) {}&= \left[
\left(-2l-2 + \rho + \frac{l(l+1)}{\rho} \right)V(\rho) + 2(l+1-\rho)\frac{\partial }{\partial \rho}V(\rho) + \rho \frac{\partial^2}{\partial \rho^2}V(\rho)
\right] \rho^l e^{-\rho}
\end{align} 
Questo ci porta ad avere l'eq di Schrödinger nella forma
\begin{align}
\rho  \frac{\partial^2}{\partial \rho^2}V(\rho) + 2(l+1-\rho)\frac{\partial }{\partial \rho}V(\rho) + (\lambda - 2l - 2)=0
\end{align}
Per equazioni differenziali di questo tipo la soluzione sarà del tipo
\begin{equation}
V(\rho)= \sum_{j=0}^\infty c_j \rho^j
\end{equation}
da cui
\begin{align}
{}&V'(\rho)= \sum_{j=1}^\infty c_j  j \rho^{j-1} \\
 &V''(\rho)= \sum_{j=2}^\infty c_j j(j-1) \rho^{j-2}
 \end{align}


Applicando i cambi di variabili
 \begin{align}
 	m=j-1 {}&\leftrightarrow j=m+1\\
 	k=j-2 &\leftrightarrow j=k+2
 \end{align} 
otteniamo
 \begin{align}
	&V'(\rho)= \sum_{m=0}^\infty c_{m+1} (m+1) \rho^{m} \\
 	&V''(\rho)= \sum_{k=0}^\infty c_{k+2} (k+2)(k+1)\rho^{k}
 \end{align}
Notiamo però che j,m,k sono dei cosidetti \textbf{indici muti}, e possiamo quindi scrivere tutto con j:
 \begin{align}
 	{}&V'(\rho)=  \sum_{j=0}^\infty c_{j+1} (j+1)\rho^j \\
 	&V''(\rho) =  \sum_{j=0}^\infty c_{j+2} (j+1)(j+2)\rho^j
 \end{align}
fatto ciò definiamo
 \begin{align}
 	a= 2(l+1-\rho) \\
 	b= \lambda - 2l - 2
 \end{align}
E possiamo quindi riscrivere l'eq di Shroedinger come
 \begin{align}
 	&\rho \sum_{j=0}^\infty c_{j+2} (j+1)(j+2)\rho^j + a\sum_{j=0}^\infty c_{j+1} (j+1)\rho^j + b\sum_{j=0}^\infty c_j \rho^j=0 \nonumber \\
 	& \downarrow  \nonumber \\
 	&\sum_{j=0}^\infty c_{j+2} (j+1)(j+2)\rho^j + \frac{a}{\rho}\sum_{j=0}^\infty c_{j+1} (j+1)\rho^j + \frac{b}{\rho}\sum_{j=0}^\infty c_j \rho^j=0
 \end{align}
 
 studiamo singolarmente i tre contributi:
 
 \begin{align}
\sum_{j=0}^\infty c_{j+2} (j+1)(j+2)\rho^j = \sum_{j=0}^\infty c_{j+2} j(j+1)\rho^{j-1} 
 \end{align}
 
 \begin{align}
\frac{a}{\rho}\sum_{j=0}^\infty c_{j+1} (j+1)\rho^j {}&= 
\frac{2l + 2}{\rho}\sum_{j=0}^\infty c_{j+1} (j+1)\rho^j - 2 \sum_{j=0}^\infty c_{j+1} (j+1)\rho^j = \nonumber \\
&= (2l + 2)\sum_{j=0}^\infty c_{j+1} (j+1)\rho^{j-1} - 2 \sum_{j=0}^\infty c_{j} j\rho^{j-1}
 \end{align}
 
 \begin{align}
	 \frac{b}{\rho}\sum_{j=0}^\infty c_j \rho^j= b\sum_{j=0}^\infty c_j \rho^{j-1}
 \end{align}
 da cui
 \begin{align}
	  \sum_{j=0}^\infty [ (2l+2+j)(j+1)c_{j+1}  + (\lambda - 2l - 2 - 2j)c_j ] \rho^{j-1}=0
 \end{align}

 le soluzioni non banali si trovano per
  \begin{align}
  (2l+2+j)(j+1)c_{j+1}  + (\lambda - 2l - 2 - 2j)c_j =0
 \end{align}
  e troviamo quindi
  \begin{align}
 c_{j+1}  = \frac{2l + 2 + 2j - \lambda}{(2l+2+j)(j+1)}c_j 
 \end{align}
 
 Questa è però un'espressione \textbf{ricorsiva}! Possiamo usarla per ricavare tutti i $c_j$. Iniziamo notando che per $j \rightarrow \infty$ :
 \begin{align}
 c_j |_{j \rightarrow \infty} \simeq \frac{2j}{j(j+1)}c_j = \frac{2}{j+1}c_j \simeq \frac{2}{j}c_j
 \end{align}
 questo significa che
 \begin{align}
c_1= \frac{2}{1}c_o = 2c_0 \rightarrow c_2= \frac{2}{2}c_1= \frac{2}{2}\frac{2}{1}c_o \rightarrow \dots
\end{align}
e per induzione
\begin{align}
c_j = \frac{2^j}{j!}c_0
\end{align}
da cui troviamo che
\begin{align}
V(\rho)= \sum_{j=0}^\infty \frac{2^j}{j!}c_0 \rho^j = \sum_{j=0}^\infty \frac{(2\rho)^j}{j!}c_0
\end{align}
Ricordando che
\begin{align}
\sum_{j=0}^\infty \frac{x^n}{n!} = e^x
\end{align}
ponendo $x= 2\rho$ ottiano infine
\begin{align}
V(\rho)= c_o e^{2\rho} \rightarrow  u_{E l}(\rho)= \rho^{l+1}e^{-\rho}V(\rho)= c_0 \rho^{l+1} e^\rho
\end{align}

C'è però un problema: per $\rho \rightarrow \infty$ la serie diverge! Questo implica che, affinché il tutto abbia un significato fisico, deve esistere un limite superiore. Presa l'espressione ricorsiva, avremo quindi un $j_{max}$ tale che
\begin{align}
2(l + j_{max} +1) - \lambda=0
\end{align}

definiamo il \textbf{numero quantico principale} come
\begin{equation}
n= j_{max} + l + 1 \qquad \rightarrow \qquad \lambda = 2n
\end{equation}

ricordiamo che $\lambda$ è definito come:
\begin{align}
\lambda = \frac{\mu q^2}{4 \pi \epsilon_0} \frac{1}{\sqrt{smE}}\frac{1}{\hbar}
\end{align}
ribaltando l'espressione otteniamo che
\begin{align}
2mE= \left( \frac{\mu q^2}{4 \pi \epsilon_0 \hbar} \frac{1}{\lambda} \right)^2 
\end{align}

da cui, ricordando che in questo caso $\mu \sim m$, ricaviamo infine i \textbf{livelli dell'energia dell'atomo di Idrogeno}, ossia gli autovalori dell'hamiltoniana:
\begin{align}
E_n= \left( \frac{\mu q^2}{8 \pi \epsilon_0 \hbar} \right)^2 \frac{1}{n^2}
\end{align}

Notiamo che lo stato fondamentale si ha per $n_{min}=1$ e che 
\begin{align}
E_1= -13.6 \; eV
\end{align}

\begin{align}
j_{max}= n-l-1 = 1-l -1=-l
\end{align}

ma $j_{max} \geq 0$ affinché $V(\rho) \neq 0$, quindi 

\begin{align}
j_{max}=l=0
\end{align}

Questo significa che l'autofunzione associata allo stato fondamentale sarà, nella forma $\psi_{nlm}$
\begin{align}
\psi_{100}= R_{10}(r)Y_0^0(\theta, \phi) 
\end{align}
presa la ricorsiva, avremo che, per $j_{max} = 0 \, , \, l=0\, , \, n=1 \, , \, \lambda=2$:
\begin{align}
 c_1 = \frac{2-2}{2}c_0=0 \rightarrow c_n=0 \quad \forall x \geq 1
\end{align}
Da cui troviamo che
\begin{align}
V(\rho)= \sum_0^{j_{max}} c_j\rho^j = c_0
\end{align}
Questo ci porta a scrivere che
\begin{align}
R_{1 0}= \frac{1}{2}\rho e^{-rho}c_0
\end{align}

Definiamo ora il \textbf{raggio di Bohr} come
\begin{align}
a= \frac{4\pi\epsilon_0 \hbar^2}{m q^2}= 5.28 \cdot 10^{-11} \, m
\end{align}
e definiamo
\begin{align}
\rho= \frac{r}{an}
\end{align}
per riscrivere quindi
\begin{align}
R_{10}(r)= \frac{c_0}{a}e^{-\frac{r}{a}}
\end{align}

Possiamo finalmente ricavare $c_0$, grazie alla normalizzazione delle funzioni d'onda e integrando per parti:
\begin{align}
\int_o^\infty dr |R_{10}(r)|^2r^2= \frac{|c_0|^2}{a^2} \int_o^\infty dr r^2 e^{-\frac{r}{a}}= \frac{|c_0|^2}{a^2}\frac{a^3}{4}= \frac{a|c_0|^2}{4}=1 
\end{align}
da cui ricaviamo quindi
\begin{align}
c_0= \frac{2}{\sqrt{a}}
\end{align}

ricordando che $Y_0^0 = \frac{1}{\sqrt{4\pi}}$ possiamo quindi scrivere l'autofunzione dello stato fondamentale come
\begin{align}
\psi_{100}= \frac{1}{\sqrt{\pi a^3}}e^{-\frac{r}{a}}
\end{align}

La situazione si complica per gli stati eccitati, vediamo ad esempio il primo stato, $n=2$:
\begin{align}
E_2 = \frac{E_1}{n^2} = \frac{E_1}{4}= -3.4 eV
\end{align}
\begin{align}
n=2 \rightarrow j_{max}= 1 - l \rightarrow \left\{
\begin{array}{cc}
j_{max} =0 \, , \, l=1 \, , \, m=0,\pm 1 \\
j_{max}=1 \, , \, l=0 \, , \, m=0 \quad \;\;\:
\end{array}
\right.
\end{align}
e quindi lo stato è 4 volte degenere, con autofunzioni
\begin{align}
&\psi_{200}(r,\theta, \phi)= R_{20}(r)Y_0^0 (\theta, \phi) \, ;\, \left\{
\begin{array}{ccc}
\psi_{21+1}(r,\theta, \phi)=R_{21}(r)Y_1^{+1}(\theta, \phi) \\
\psi_{21 0}(r,\theta, \phi)=R_{21}(r)Y_1^0 (\theta, \phi)\quad\\
\psi_{21-1}(r,\theta, \phi)=R_{21}(r)Y_1^{-1}(\theta, \phi) \\
\end{array}
\right. \\
&\nonumber \\
&\nonumber \\
&j_{max} =1 \, , \, l=0 \rightarrow c_1= \frac{2-4}{2}c_0=-c_0 \rightarrow V(\rho)= c_o(1 -\rho) \\
&R_{20}= \frac{c_0}{2a}e^{-\frac{r}{2a}} \\
&\nonumber \\
&\nonumber \\
&j_{max} =0 \, , \, l=1 \rightarrow c_1= \frac{4-4}{2}c_0=-0 \rightarrow V(\rho)= c_o\\
&R_{21}= \frac{c_0 r}{4a^2}e^{-\frac{r}{2a}}
\end{align}
\newpage
Per i casi successivi il problema può essere scritto in modo più elegante e compatto tramite i \textbf{polinomi di Laguerre}:

\begin{align}
{}&L_q(x)= \frac{e^x}{n!}\frac{\partial^n}{\partial x^n}(e^{-x}x^n)\\
&L_{q-p}^p=(-1)^p \left( \frac{\partial}{\partial x} \right)^p L_q(x) \\
&V(\rho)= L_{n-l-1}^{2l+1}(2\rho)
\end{align}