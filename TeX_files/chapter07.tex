\chapter{Teoria delle perturbazioni}

La teoria delle perturbazioni è uno strumento molto potente nello studio di sistemi complessi. Il procedimento si basa sullo scomporre il sistema studiato in un caso noto (ex. l'Oscillatore Armonico) per poi aggiungere mano a mano piccoli contributi fino ad approssimare quanto si vuole il sistema da studiare.

\section{Perturbazioni indipendenti dal tempo}

Supponiamo di avere
\begin{align}
H^0\psi_n^0=E_n^0\psi_n^0 \quad ; \quad \braket{\psi_m^0|\psi_n^0}=\delta_{mn}
\end{align}
Immaginiamo ora di sommare un piccolo contributo. Avremo
\begin{align}
H\psi_n=E_n\psi_n 
\end{align}
con
\begin{align}
{}&H= H^0 + \lambda H'\\
&\lim_{\lambda \rightarrow 0}H= H^0
\end{align}

Espandiamo in serie le funzioni d'onda e gli autovalori dell'energia in funzione di $\lambda$:
\begin{align}
{}&\psi_n= \psi^0_n + \lambda \psi_n^1+ \lambda^2 \psi_n^2+\dots \\
&E_n= E_n^0 + \lambda E_n^1+ \lambda^2 E_n^2+\dots
\end{align}

Questo ci porta a riscrivere, 
\begin{align}
{}&H^0\psi^0_n=E^0_0n\psi^0_n	\qquad \text{eq. imperturbata}\\
&\lambda(H^0\psi'_n + H'\psi^0_n)=\lambda(E'_n \psi^0_n + E^0_n\psi'_n) \qquad\qquad\qquad\qquad\qquad\,\, \text{I ordine}\\
&\lambda^2 (H^0\psi^2_n + H'\psi'_n)=\lambda^2 (E^0_n \psi^2_n + E^0_n \psi_n^2 + E_n^2\psi_n^0 + E_n'\psi_n')\qquad \text{II ordine} \\
\vdots \nonumber
\end{align}

\subsection{Perturbazione al I ordine}

\begin{align}
H^0\psi_n' + H'\psi_n^0= E_n'\psi_n^0 + E_n^0\psi_n' 
\end{align}

Moltiplicando entrambi i lati per il complesso coniugato di $\psi_n^0$ possiamo scrivere, passando in notazione di Dirac:
\begin{align}
\braket{\psi_n^0 |H^0|\psi_n'} + \braket{\psi_n^0 |H'|\psi_n^0} = E_n'\braket{\psi_n^0|\psi_n^0} + E_n^0 \braket{\psi_n^0|\psi_n'}
\end{align}

Siccome non sappiamo nulla delle $\psi_n'$ possiamo supporre che essa sia ortogonale a $\psi_n^0$, e quindi, ricordando che H è hermitiano:
\begin{align}
E_n^0 \, \cancel{\braket{\psi_n^0 |\psi_n'}}+ \braket{\psi_n^0 |H'|\psi_n^0}= E_n' + E_n^0 \, \cancel{\braket{\psi_n^0 |\psi_n'}}
\end{align}
ossia
\begin{align}
E_n'=\braket{\psi_n^0 |H'|\psi_n^0}
\end{align}

ed abbiamo così trovato la correzione dell'energia  al primo ordine.

Che forma avranno invece le correzioni alle autofunzioni?

Ricordiamo che se le $\psi_n^0$ formano un set di autofunzioni completo possiamo riscrivere le $\psi_n'$ come combinazioni lineari di queste:
\begin{align}
\psi_n'= \sum_{m\neq n}c_m^{(n)}\psi_m^0
\end{align}

La condizione $m\neq n$ è necessaria in quanto se $\psi_n'$ è soluzione dell'equazione sicuramente lo sarà la combinazione $\psi_n' + \alpha \psi_n^0$ con $\alpha=cost.$

\newpage

Questo ci porta ad avere l'eq.di Schrödinger nella forma
\begin{align}
(H^0 - E^0_n)\ket{\psi_n'} = (H^0 - E^0_n)\sum_{m\neq n}c_m^{(n)}\ket{\psi_m^0} = -(H'-E')\ket{\psi_n^0}
\end{align}
che possiamo riscrivere
\begin{align}
\sum_{m\neq n}c_m^{(n)} (E_m^0 - E^0_n) \ket{\psi_m^0} = -(H'-E')\ket{\psi_n^0}
\end{align}
Se ora moltiplichiamo per una generica $\ket{\psi_k^0}$ otteniamo
\begin{align}
\sum_{m\neq n}c_m^{(n)} (E_m^0 - E^0_n) \braket{\psi_k^0|\psi_m^0} = -(H'-E')\braket{\psi_k^0|\psi_n^0}
\end{align}
Notiamo che se $k=n$ il termine a sinistra si annulla, quindi deve essere $k\neq n$, da cui
\begin{align}
\sum_{m\neq n}c_m^{(n)} (E_m^0 - E^0_n)\delta_{km} = -\braket{\psi_k^0|H'|\psi_n^0} +E'\, \cancel{\braket{\psi_k^0|\psi_n^0}}
\end{align}
che diventa
\begin{align}
{}&c_m^{(n)} (E_m^0 - E^0_n) = -\braket{\psi_k^0|H'|\psi_n^0}\\
&\downarrow \nonumber \\
\label{eq:eq0}
& c_m^{(n)}  = \frac{\braket{\psi_k^0|H'|\psi_n^0}}{E_n^0 - E^0_m}
\end{align}
da cui otteniamo infine
\begin{align}
\psi_n'= \sum_{m\neq n} \frac{\braket{\psi_k^0|H'|\psi_n^0}}{E_n^0 - E^0_m} \psi_m^0
\end{align}

\subsection{Perturbazione al II ordine}

Ricordiamo che al II ordine abbiamo:
\begin{align}
         \label{eq:eq1}
H^0\psi^2_n + H'\psi'_n=E^0_n \psi^2_n + E^2_n \psi_n^0 + E_n^2\psi_n^0 + E_n'\psi_n'
\end{align}

Avremo di nuovo:
\begin{align}
\label{eq:eq2}
\psi_n'= \sum_{m\neq n}c_m^{(n)}\psi_m^0
\end{align}

ma stavolta dobbiamo tenere da conto anche $\psi_n^2$, di cui non sappiamo nulla.

Partiamo di nuovo dalle correzioni sull'energia. Passiamo in notazione di Dirac per comodità e moltiplichiamo da entrambi i lati per $\bra{\psi_n^0}$ la $\eqref{eq:eq1}$. Ricordando le considerazioni del paragrafo scorso otteniamo 
\begin{align}
\label{eq:eq3}
{}& \braket{\psi_n^0|H^0|\psi^2_n} + \braket{\psi_n^0|H'|\psi'_n}=E^0_n\cancel{\braket{\psi_n^0|\psi^2_n}} + E^2_n\braket{\psi_n^0|\psi_n^0} + E_n'\cancel{\braket{\psi_n^0|\psi_n'}}\nonumber \\
& \downarrow \\
& E_n^0\cancel{\braket{\psi_n^0|\psi^2_n}} + \braket{\psi_n^0|H'|\psi'_n}= E^2_n\braket{\psi_n^0|\psi_n^0}\nonumber \\
& \downarrow \\
& E^2_n=\braket{\psi_n^0|H'|\psi'_n}
\end{align}

Vista la $\eqref{eq:eq2}$ e la $\eqref{eq:eq0}$ possiamo scrivere
\begin{align}
E^2_n{}&=\sum_{m\neq n} c_m\braket{\psi_n^0|H'|\psi^0_m} \\
&= \sum_{m\neq n}\frac{\braket{\psi_m^0|H'|\psi_n^0} \braket{\psi_n^0|H'|\psi^0_m}}{E_n^0 - E^0_m}
\end{align}

Notiamo come il denominatore non crei problemi fintantoché lo spettro energetico degli stati imperturbati è non degenere.

Siccome $|x^2|= x^* x$ e $\braket{\psi_m^0|H'|\psi_n^0}= (\braket{\psi_n^0|H'|\psi^0_m})^*$ possiamo scrivere in modo più compatto
\begin{align}
E^2_n= \sum_{m\neq n}\frac{|\braket{\psi_m^0|H'|\psi_n^0}|^2}{E_n^0 - E^0_m}
\end{align}

\newpage

\subsection{Caso degenere}

Affrontiamo ora il problema della degenerazione.
Avremo, dato un livello dell'energia imperturbata
\begin{align}
E_d^0 \; t.c. \left\{\begin{array}{cc}
H^0\ket{\psi^0_a}= E_d^0\ket{\psi^0_a} \\
H^0\ket{\psi^0_b}= E_d^0\ket{\psi^0_b}
\end{array}
\right. \quad ; \quad \braket{\psi_a^0 | \psi_b^0}=0
\end{align}

Sappiamo già che anche $\ket{\psi^0}= \alpha \ket{\psi^0_a} + \beta \ket{\psi^0_b}$ sarà autofunzione di $H$ tale che $H^0\ket{\psi^0}=E_d^0\ket{\psi^0}$.

Cosa accade quando si perturba l'energia?

Nel migliore dei casi si ha una rimozione della degenerazione, e che quando "spegniamo" la perturbazione $H'$ (cioè si ha $\lambda \rightarrow 0$)avremo che
\begin{enumerate}
	\item lo stato ad energia più alta cadrà in una combinazione $\ket{\psi_1}$ tra $\ket{\psi^0_a}$ e $\ket{\psi^0_b}$
	\item lo stato ad energia più bassa cadrà in una combinazione
	$\ket{\psi_2}\neq \ket{\psi_1}$
\end{enumerate} 

Iniziamo dal punto di partenza del caso non degenere. Al I ordine avremo
\begin{align}
H^0\ket{\psi'} + H'\ket{\psi^0}=E^0\ket{\psi'} + E'\ket{\psi^0}
\end{align}

moltiplichiamo tutto per $\bra{\psi_a^0}$ e otteniamo
\begin{align}
{}&\braket{\psi_a^0|H^0|\psi^1} + \braket{\psi_a^0|H'|\psi^0} = E^0 \cancel{\braket{\psi_a^0|\psi'}} + E'\braket{\psi_a^0|\psi^0}\\
&\downarrow \nonumber \\
&E^0\cancel{\braket{\psi_a^0|\psi^1}} + \braket{\psi_a^0|H'|\psi^0} =  E'\braket{\psi_a^0|\psi^0}\\
&\downarrow \nonumber \\
&\braket{\psi_a^0|H'|\psi^0} =  E'\braket{\psi_a^0|\psi^0}\\
&\downarrow \nonumber \\
&\bra{\psi_a^0}H'(\alpha \ket{\psi^0_a} + \beta \ket{\psi^0_b})= E'\bra{\psi_a^0}(\alpha \ket{\psi^0_a} + \beta \ket{\psi^0_b}) \\
&\downarrow \nonumber \\
&\alpha \braket{\psi_a^0 |H'|\psi_a^0} + \beta\braket{\psi_a^0 |H'|\psi_b^0} = E' ( \alpha \braket{\psi_a^0|\psi_a^0} + \beta \cancel{\braket{\psi_a^0|\psi_b^0}} )
\end{align}
chiamiamo
\begin{align}
W_{aa}= \braket{\psi_a^0 |H'|\psi_a^0} \\
W_{ab}= \braket{\psi_a^0 |H'|\psi_b^0}
\end{align}
e riscriviamo così
\begin{align}
\alpha E' = \alpha W_{aa} + \beta W_{ab}
\end{align}
svolgendo gli stessi conti con $\bra{\psi_b^0}$
\begin{align}
{}&W_{ba}= \braket{\psi_b^0 |H'|\psi_a^0} \\
&W_{bb}= \braket{\psi_b^0 |H'|\psi_b^0}\\
&\beta E' = \alpha W_{ba} + \beta W_{bb}
\end{align}
moltiplichiamo $E'$ nella seconda per $W_{ab}$ ottenendo
\begin{align}
{}&\beta E' W_{a_b} = \alpha W_{ba}W_{a_b} + \beta W_{bb}W_{a_b} \\
&\downarrow \nonumber \\
&E' \beta  W_{a_b} = \alpha W_{ba}W_{a_b} + \beta W_{bb}W_{a_b}
\end{align}
e sostituiamo, ricavandolo dall'altra, $\beta W_{ab}= \alpha E' - \alpha W_{aa}$ ottenendo
\begin{align}
{}&E' \alpha E' - E'\alpha W_{aa} = \alpha W_{a_b}W_{ba} + \beta W_{a_b}W_{bb} \\
&\downarrow \nonumber \\
&\alpha W_{a_b}W_{ba} + W_{bb}[\alpha E' - \alpha W_{aa}] - E'[\alpha E' - \alpha W_{aa}]=0 \\
&\downarrow \nonumber \\
& \alpha [W_{a_b}W_{ba} - (E'-W_{bb})(E' - W_{aa})] =0
\end{align}

\newpage

\textbf{Qualora $\alpha \neq 0$}:
\begin{align}
(E')^2 - E'(W_{aa} + W_{bb}) - (W_{aa}W_{bb} - W_{ab}W_{ba})=0
\end{align}
Abbiamo così un'equazione lineare al II ordine, le cui soluzioni saranno
\begin{align}
E'_\pm = \frac{1}{2} [W_{aa} + W_{bb} \pm \sqrt{(W_{aa}-W_{bb})^2 + 4|W_{ab}|^2}]
\end{align}
Abbiamo così trovato le correzioni all'energia.

Cosa succede se uno dei due coefficienti è nullo? Semplicemente ci si riconduce al caso non degenere.

\section{Perturbazioni dipendenti dal tempo (incomplete)}

Studiamo il problema nel caso di un \textbf{sistema a due livelli}. Avremo solo due livelli energetici possibili per l'hamiltoniana imperturbata:
\begin{align}
{}&\left\{\begin{array}{cc}
H^0\ket{\psi^0_a}= E_a^0\ket{\psi^0_a} \\
H^0\ket{\psi^0_b}= E_b^0\ket{\psi^0_b}
\end{array}
\right. \quad;\quad \braket{\psi_a^0 |\psi_b^0}= \delta_{ab} \\
\nonumber \\
&\ket{\psi^0}= c_a\ket{\psi^0_a} + c_b\ket{\psi^0_b}
\end{align}
In assenza di perturbazioni, in caso di evoluzione temporale essi evolvono indipendentemente:
\begin{align}
{}&\ket{\psi^0}= c_a e^{-i \frac{E_a}{\hbar}t} \ket{\psi^0_a} + c_b e^{-i \frac{E_b}{\hbar}t} \ket{\psi^0_b} \\
&|\psi^0|^2= |c_a|^2 + |c_b|^2=1
\end{align}
Aggiungiamo una perturbazione $H'(t)$ dipendente dal tempo. Le autofunzioni dovranno rimanere uguali, correzioni permettendo, ma avremo dei coefficienti dipendenti dal tempo, dato che se scriviamo $\psi(r,t) = \psi(r)e^{-i \frac{E}{\hbar}t}$ nell'equazione di Schroedinger $V(r,t)$ agirà solo su $\psi(r)$, e il risultato di tale applicazione saranno proprio i coefficienti.

Cerchiamo quindi una soluzione del tipo
\begin{align}
\ket{\psi^0}= c_a(t) e^{-i \frac{E_a}{\hbar}t} \ket{\psi^0_a} + c_b(t) e^{-i \frac{E_b}{\hbar}t} \ket{\psi^0_b}
\end{align}
Ricordando che
\begin{align}
&H= H^0 + H'(t) \\
&H\psi = i \hbar \frac{\partial}{\partial t} \psi
\end{align}
Possiamo riscrivere
\begin{align}
&(H^0 + H'(t))(c_a(t) e^{-i \frac{E_a}{\hbar}t} \psi^0_a + c_b(t) e^{-i \frac{E_b}{\hbar}t} \psi^0_b)=  i \hbar \frac{\partial}{\partial t} \psi^0\\
\downarrow \nonumber \\
& c_a(t) e^{-i \frac{E_a}{\hbar}t} H^0\psi^0_a + c_a(t) e^{-i \frac{E_a}{\hbar}t}H'(t)\psi^0_a +  c_b(t) e^{-i \frac{E_b}{\hbar}t} H^0\psi^0_b + c_b(t) e^{-i \frac{E_b}{\hbar}t}H'(t)\psi^0_b=  \nonumber \\
&=i\hbar \left[ \dot{c_a}\psi^0_a e^{-i \frac{E_a}{\hbar}t} -i\frac{E_a}{\hbar}c_a\psi^0_a e^{-i \frac{E_a}{\hbar}t} + 
\dot{c_b}\psi^0_b e^{-i \frac{E_b}{\hbar}t} -i\frac{E_b}{\hbar}c_a\psi^0_b e^{-i \frac{E_b}{\hbar}t}
\right]\\
&c_a(t) e^{-i \frac{E_a}{\hbar}t} E_a^0\psi^0_a + c_a(t) e^{-i \frac{E_a}{\hbar}t}H'(t)\psi^0_a +  c_b(t) e^{-i \frac{E_b}{\hbar}t} E_b^0\psi^0_b + c_b(t) e^{-i \frac{E_b}{\hbar}t}H'(t)\psi^0_b=  \nonumber \\
&= i\hbar \left[
\dot{c_a}\psi^0_a e^{-i \frac{E_a}{\hbar}t} + \dot{c_b}\psi^0_b e^{-i \frac{E_b}{\hbar}t}
\right]
+ c_a(t) e^{-i \frac{E_a}{\hbar}t} E_a^0\psi^0_a +  c_b(t) e^{-i \frac{E_b}{\hbar}t} E_b^0\psi^0_b \nonumber \\
&\downarrow \nonumber \\
&c_a(t) e^{-i \frac{E_a}{\hbar}t}H'(t)\psi^0_a + c_b(t) e^{-i \frac{E_b}{\hbar}t}H'(t)\psi^0_b= i\hbar \left[
\dot{c_a}\psi^0_a e^{-i \frac{E_a}{\hbar}t} + \dot{c_b}\psi^0_b e^{-i \frac{E_b}{\hbar}t}
\right]
\end{align}

Proviamo ora ad isolare i due coefficienti.

Iniziamo passando in notazione di Dirac e prima moltiplicando per $\bra{\psi_a^0}$:
\begin{align}
c_a e^{-i \frac{E_a}{\hbar}t} \braket{\psi_a^0|H'|\psi_a^0} + 
c_b e^{-i \frac{E_b}{\hbar}t} \braket{\psi_a^0|H'|\psi_b^0} = i \hbar \dot{c_a} e^{-i \frac{E_a}{\hbar}t}
\end{align}

definiamo ora per pulizia $H'_{ij} = \braket{\psi_i|H'|\psi_j}$ e $\omega_{ij}=E_i - E_j$. 

Ci rendiamo conto di essere di fronte ad un'equazione differenziale:
\begin{align}
\dot{c_a} = -\frac{i}{\hbar}[H'_{aa} c_a + H'_{ab} e^{i\frac{\omega_{ab}}{\hbar}t} c_b]
\end{align}
In modo più compatto per casi più generali:
\begin{align}
\dot{c_m}(t)= -\frac{i}{\hbar}\sum_n H'_{mn} e^{i\frac{\omega_{mn}}{\hbar}t}c_n
\end{align}

ROBA DA CHIEDERE POCO CHIARA, APPUNTI PERSI

\subsection{Oscillazioni armoniche (mancanti)}

Supponiamo di avere una dipendenza temporale della forma

\begin{align}
H'(r,t)= V(r) \cos{\omega t} \rightarrow H'_{ab}= \braket{\psi_a|V|\psi_b} \cos{\omega t} = V_{ab}\cos{\omega t}
\end{align}

Supponiamo che gli elementi diagonali siano nulli.

*APPUNTI PERSI*