\chapter{Il momento angolare}


\section{Introduzione}

Il momento angolare viene definito come

\begin{equation}
\vec{\mathbf{M}}= \vec{\mathbf{q}} \wedge \vec{\mathbf{p}}
\end{equation}

La sua utilità risiede, fra le altre cose, in due importanti utilizzi

\begin{enumerate}
	
	\item Fornisce importanti informazioni sul comportamento dei corpi in rotazione
	
	\item	La sua conservazione è indice di una simmetria del sistema rispetto alle rotazioni, il che ci permette di studiare simili sistemi lungo qualunque asse preferiamo.
	\end{enumerate}

\section{Principio di conservazione}

Come si definisce in meccanica quantistica il concetto di conservazione?

Sia un'osservabile rappresentata da f(q,t), e studiamone l'evoluzione temporale

\begin{equation}
\frac{d}{dt}\braket{f} = \frac{d}{dt} \int dq \:\: \psi^* \, \hat f \: \psi = 
\int dq \:\: \psi^* \: \frac{\partial \hat f}{\partial t} \: \psi + 
\int dq \:\: \frac{\partial \psi^*}{\partial t} \hat f \psi + 
\int dq \:\: \psi^* \, \hat f \: \frac{\partial \psi}{\partial t}
\end{equation}

dal capitolo precedente notiamo che

\begin{equation}
\frac{\partial \psi}{\partial t} = -\frac{i}{\hbar}H\psi
\: \leftrightarrow \:
\frac{\partial \psi^*}{\partial t} = \frac{i}{\hbar}H^*\psi*
\end{equation}

Sostuiamo nella (3.2)

\begin{equation}
\frac{d}{dt}\braket{f} =  
\int dq \:\: \psi^* \: \frac{\partial \hat f}{\partial t} \: \psi + 
\frac{i}{\hbar} \int dq \:\: H^*\psi* \hat f \psi - 
\frac{i}{\hbar} \int dq \:\: \psi^* \, \hat f \: H\psi
\end{equation}

che diventa

\begin{equation}
\frac{d}{dt}\braket{f} =  
\int dq \:\: \psi^* \left[ \frac{\partial \hat f}{\partial t}
+ \frac{i}{\hbar} \left( H^* f - f H \right) \right]
\end{equation}

Ricordando che H è hermitiano, otteniamo che se

\begin{equation}
H^* f - f H = Hf - fH = 0 \rightarrow [H,f]=0 
\end{equation}

avremo

\begin{equation}
\frac{d}{dt}\braket{f} = 0 \\ 
\end{equation}

E quindi un'osservabile si conserva se è compatibile con una misura dell'energia.
\newpage

\section{Invarianza per rotazioni e conservazione del momento angolare}

Sia uno spazio $(x,y,z)$, e sia un punto $(x_p \, , y_p\, , z_p)$. Se ruotiamo il piano (x,y) intorno z di un angolo $\phi$, generiamo la seguente trasformazione:


\begin{align}
\left\{\begin{array}{ccc}
x_p' ={}& x_p \cos(\phi) + y_p \sin(\phi) \\ 
y_p' = & y_p \cos(\phi) - x_p \sin(\phi) \\
z_p' = & z_p \qquad \qquad \qquad \qquad \\
\end{array}\right. 
\end{align}

con matrice unitaria di rotazione

\begin{equation}
\left(
 \begin{array}{ccc}

\cos(\phi) & \sin(\phi) \\
-\sin(\phi) & \cos(\phi) 
 
\end{array}  
\right)
\end{equation}

Se $\phi \rightarrow 0$ abbiamo che

\begin{equation}
\left\{\begin{array}{ccc}

\cos(\phi) = 1 \\ 
\sin(\phi) = \delta \phi

\end{array}\right. 
\end{equation}


E le (3.8) e (3.9) diventano

\begin{equation}
\left\{\begin{array}{ccc}

x_p'={}& x_p + y_p \, \delta  \phi \\ 
y_p' =& y_p - x_p \, \delta  \phi \\
z_p' =& z_p \qquad \quad

\end{array}\right. 
\end{equation}

\begin{equation}
\left(
\begin{array}{ccc}

1 & \delta \phi \\
-\delta \phi & 1 

\end{array}  
\right)
\end{equation}

Come si comporterà la funzione d'onda che descrive lo stato in $(x_p \, , y_p\, , z_p)$?

\begin{equation}
\psi(x,y,t) \rightarrow \psi'(x',y',z')= \psi(x_p + y_p \, \delta  \phi , y_p - x_p \, \delta  \phi, z)
\end{equation}

La funzione può essere espressa come variazioni delle singole componenti spaziali:


\begin{equation}
\begin{split}
\psi'(x',y',z'){}&= \psi(x,y,z) + \frac{\partial \psi}{\partial x} \, \delta \phi \, y - \frac{\partial \psi}{\partial y}\, \delta \phi \, x = \\
&= \psi(x,y,z) +  \delta \phi \left[ y \, \frac{\partial \psi}{\partial x}   - x \, \frac{\partial \psi}{\partial y} \right] = \\
&= \psi(x,y,z) +  \delta \phi \left[ y \, \frac{\partial }{\partial x}  - x \, \frac{\partial }{\partial y} \right]\psi = \\
&= \left[ 1 + \delta \phi \left( y \partial x - x \partial y \right)  \right]\psi(x,y,z)
\end{split}
\end{equation}

Questo significa che la trasformazione è generata da un operatore $\mathbf{R_0}$, definito come

\begin{equation}
\mathbf{R_0} =  1 + \delta \phi \left( y \partial x - x \partial y \right)
\end{equation}

Come troviamo la sua definizione in notazione di Dirac? \newline

Ricordiamo che il momento angolare può essere scritto come 

\begin{equation}
\mathbf{\hat L} = \mathbf{\hat r} \wedge \mathbf{\hat p}
\end{equation}

da cui
\begin{align}
\begin{split}
\mathbf{\hat L} {}& = \mathbf{\hat r} \wedge \mathbf{\hat p} = 
\left|\begin{array}{ccc}\hat i & \hat j & \hat k \\ x & y & z \\ p_x & p_y & p_z \end{array}\right| =  \\
&= \hat i [y p_z - zp_y] - \hat j [x p_z - zp_x] + \hat k [x p_y - yp_x]
\end{split}
\end{align}

\newpage

Siccome abbiamo anche che

\begin{align}
\begin{split}
\mathbf{\hat L} {}&= \mathbf{\hat L_x} + \mathbf{\hat L_y}  + \mathbf{\hat L_z}= \\
& = \hat i L_x + \hat j L_y + \hat k L_z 
\end{split}
\end{align}

Possiamo ricavare

\begin{align}
L_x= y p_z - z p_y \\
L_y= z p_x - x p_z \\
L_z= x p_y - y p_x
\end{align}

Scelta la rotazione lungo l'asse z, prendiamo in considerazione tale componente di $\mathbf{\hat L}$, e ricordando la definizione in MQ dell'impulso otteniamo

\begin{equation}
L_z = x p_y - yp_x = i\hbar [y \partial x - x \partial y]
\end{equation}

prendiamo ora la (3.15) e vi sostituiamo la (3.18), ottenendo

\begin{equation}
\mathbf{\hat R_0} =  1 - \frac{i}{\hbar} \, \delta \phi \, \mathbf{\hat L_z}  
\end{equation}

Quindi la rotazione viene generata dal momento angolare. Qualora esso sia costante lo stato non cambia per rotazioni nello spazio.


In caso di rotazioni non infinitesime semplicemente si seziona in rotazioni infinitesime.

ossia avremo

\begin{equation}
\mathbf{\hat R_0} \: \psi =   \mathbf{\hat r_0}\mathbf{\hat r_1} \dots \mathbf{\hat r_N} \: \psi= \prod_{n=0}^{N} \mathbf{\hat r_n} \: \psi
\end{equation}

da cui si ottiene

\begin{equation}
R_0 (\phi) = \lim_{\substack{ \phi \rightarrow 0 \\ n \rightarrow + \infty}} \left( 
1 - \frac{i}{\hbar} \, \delta \phi \, L_z
 \right)^n
 \end{equation}

La (3.21) può essere interpretata come un'approssimazione di Taylor dell'esponenziale, e possiamo quindi esprimerla come

\begin{equation}
R(\phi) = e^{- \frac{i}{\hbar} \, \delta \phi \, L_z}
\end{equation}

e quindi per concludere possiamo riscrivere la (3.14) nel seguente modo: 

\begin{equation}
\psi ' (x',y',z')= R(\psi)\psi (x,y,z)= e^{- \frac{i}{\hbar} \, \delta \phi \, L_z} \psi (x,y,z)
\end{equation}

Se andiamo a studiare la probabilità dello stato notiamo un fenomeno interessante: la probabilità è invariante per rotazioni! Infatti:

\begin{equation}
|\psi ' (x',y',z')|^2 = e^{ \frac{i}{\hbar} \, \delta \phi \, L_z}e^{- \frac{i}{\hbar} \, \delta \phi \, L_z}|\psi (x,y,z)|^2 = |\psi (x,y,z)|^2
\end{equation}

In conclusione possiamo esprimere il momento angolare nel seguente modo


\begin{equation}
\left\{\begin{array}{ccc}

L_x= y p_x - z p_y \\
L_y= x p_z - z p_x \\
L_z= x p_y - y p_x

\end{array}\right. 
\rightarrow
\left\{\begin{array}{ccc}

L_x= i\hbar[y \partial x - z \partial y]\\
L_y= i\hbar[x \partial z - z \partial x] \\
L_z= i\hbar[x \partial y - y \partial x]
\end{array}\right. 
\rightarrow \mathbf{\hat L} = i \hbar \: \mathbf{\hat r} \wedge \mathbf{\hat \nabla}
\end{equation}

\newpage

\section{Regole di commutazione del momento angolare}

\subsection{Proprietà dei commutatori}

\begin{align}
[A + B, C] &= [A,C] + [B,C] \\
[AB,C] &= A[B,C] + [A,C]B \\
[A,BC] &= [A,B]C + B[A,C] \\
[A,B] &= -[B,A]
\end{align}

\subsection{Commutazione tra q e p}

Siano le componenti di q e p corrispondenti allo stesso asse:

\begin{align}
[\mathbf{x}, \mathbf{p_x}]\psi(x) {}&= (\mathbf{x}\mathbf{p} - \mathbf{p}\mathbf{x}) = \nonumber \\ 
& = -i \hbar \left[
\mathbf{\hat x} \frac{\partial \psi}{\partial x} - \frac{\partial }{\partial x}(x \psi(x))
\right]=  \nonumber \\
&=  -i \hbar \left(
\mathbf{x} \frac{\partial \psi}{\partial x} - \psi(x) - \mathbf{x} \frac{\partial \psi}{\partial x}
\right)=  \nonumber \\
&= i \hbar \psi(x)
\end{align}

Quindi in generale avremo

\begin{equation}
[\mathbf{q_i}, \mathbf{p_i}]= i \hbar
\end{equation}

Se invece sono diverse si ottiene

\begin{align}
[\mathbf{x}, \mathbf{p_y}]\psi(x)= \dots =  -i \hbar \left(
\mathbf{x} \frac{\partial \psi}{\partial y} - 0 - \mathbf{x} \frac{\partial \psi}{\partial y}
\right)= 0
\end{align}

In modo più compatto possiamo scrivere

\begin{align}
[\mathbf{q_i}, \mathbf{p_i}]= i \hbar \delta_{ij}
\end{align}

\begin{align}
 \delta_{ij} =
\left\{\begin{array}{ccc}
1 \:\:\: i=j\\
0 \:\:\: i \neq j
\end{array}\right. 
\end{align}


\subsection{Commutazione tra q e L}

	
	Ricordando le regole del paragrafo precedente avremo che
		\begin{align}
	[\mathbf{\hat x}, \mathbf{\hat L_x}] {}&= \mathbf{\hat x}\mathbf{\hat L_x} - \mathbf{\hat L_x}\mathbf{\hat x} = \nonumber \\
	&= \mathbf{\hat x}[\mathbf{\hat y} \mathbf{\hat p_z} - \mathbf{\hat p_z} \mathbf{\hat y}] - [\mathbf{\hat y} \mathbf{\hat p_z} - \mathbf{\hat p_z} \mathbf{\hat y}]\mathbf{\hat x}=  \nonumber \\
	&= \mathbf{\hat x}[\mathbf{\hat y} , \mathbf{\hat p_z}] - [\mathbf{\hat y} , \mathbf{\hat p_z}]\mathbf{\hat x}   = 0  \\
\nonumber	\\
\nonumber	\\
			[\mathbf{\hat x}, \mathbf{\hat L_y}] {}&= \mathbf{\hat x}\mathbf{\hat L_y} - \mathbf{\hat L_y}\mathbf{\hat x} = 		 \nonumber \\
		&= \mathbf{\hat x}[\mathbf{\hat z} \mathbf{\hat p_x} - \mathbf{\hat p_x} \mathbf{\hat z}] - [\mathbf{\hat z} \mathbf{\hat p_x} - \mathbf{\hat p_z} \mathbf{\hat x}]\mathbf{\hat x}= 			 \nonumber \\ 
		&= \mathbf{\hat z} (\mathbf{\hat x}\mathbf{\hat p_x})- \mathbf{\hat x}(\mathbf{\hat x}\mathbf{\hat p_z}) - \mathbf{\hat z}(\mathbf{\hat p_x}\mathbf{\hat x}) + \mathbf{\hat x}(\mathbf{\hat p_z}\mathbf{\hat x})= \nonumber \\
		&= \mathbf{\hat z}(\mathbf{\hat x}\mathbf{\hat p_x} - \mathbf{\hat p_x}\mathbf{\hat x}) = \mathbf{\hat z}[\mathbf{\hat x},\mathbf{\hat p_x}]=
		 i \hbar \mathbf{\hat z}\\
\nonumber	\\
\nonumber	\\
		 [\mathbf{\hat x}, \mathbf{\hat L_z}] {}&=  \mathbf{\hat x}\mathbf{\hat L_z} - \mathbf{\hat L_z}\mathbf{\hat x}= \nonumber \\
		&= \mathbf{\hat x}[\mathbf{\hat x}\mathbf{\hat p_y} - \mathbf{\hat p_y}\mathbf{\hat x}] - [\mathbf{\hat x}\mathbf{\hat p_y} - \mathbf{\hat p_y}\mathbf{\hat x}]\mathbf{\hat x}= \nonumber \\
		&= \mathbf{\hat x}(\mathbf{\hat x}\mathbf{\hat p_y}) - \mathbf{\hat y}(\mathbf{\hat x}\mathbf{\hat p_x}) - \mathbf{\hat x}(\mathbf{\hat x}\mathbf{\hat p_y}) + \mathbf{\hat y}(\mathbf{\hat p_x}\mathbf{\hat x})= \nonumber \\
		&= -\mathbf{\hat p_y}(\mathbf{\hat x}\mathbf{\hat p_x} - \mathbf{\hat p_x}\mathbf{\hat x})= -\mathbf{\hat p_y}[\mathbf{\hat x},\mathbf{\hat p_x}]= -i \hbar \mathbf{\hat p_y}
		\end{align}

Possiamo scrivere il tutto in modo più compatto introducendo il \textbf{simbolo di Ricci} (anche noto impropriamente come \textbf{tensore di Levi-Civita}):

\begin{align}
\epsilon_{i,j,k}=
\left\{\begin{array}{ccc}
{}&1 \qquad i \neq j \neq k \neq i \quad i,j,k \: \: \text{ciclici} \:\: (123,231,312) \quad \: \\
&-1 \qquad i \neq j \neq k \neq i \quad i,j,k \: \: \text{non ciclici} \:\: (132,321,213) \\
&0 \qquad \text{almeno due indici uguali} \qquad \qquad \qquad \qquad \quad
\end{array}\right. 
\end{align}

(Nota: per tradizione gli indici i,j,k rappresentano rispettivamente i tre assi x,y,z)\newline

e scrivendo così

\begin{equation}
[\mathbf{\hat q_i}, \mathbf{\hat L_j}]= i \hbar \epsilon_{i,j,k} \mathbf{\hat q_k}
\end{equation}

\subsection{Commutazione tra p e L}

In modo assolutamente identico si dimostra che

\begin{equation}
[\mathbf{\hat p_i}, \mathbf{\hat L_j}]= i \hbar \epsilon_{i,j,k} \mathbf{\hat p_k}
\end{equation}
\newpage

\subsection{Commutazione tra componenti di L}

\begin{align}
[\mathbf{\hat L_x}, \mathbf{\hat L_y} ]{}&= [ (y p_z - zp_y) , (zp_x - xp_z) ] = \nonumber \\
&= [y p_z, zp_x] + [z p_y , x p_z] - [zp_y, zp_x] - [yp_z, xp_z]
\end{align}

Questo sistema ha un problema: è un processo lungo e laborioso. Procediamo invece nel seguente modo:

\begin{align}
[\mathbf{\hat L_x}, \mathbf{\hat L_y} ]{}&= [ L_x , (zp_x - xp_z) ] = \nonumber \\
&= L_x(zp_x - xp_z) - (zp_x - xp_z)L_X= \nonumber \\
&= (L_x z - zL_x)p_x - x(L_x p_z - p_z L_x)= \nonumber \\
&= [L_x, z]p_x - x[L_x, p_z]= \nonumber \\
&= x[p_z, L_x] - [z, L_x]p_x = i\hbar (x p_y - y p_x)= i \hbar \mathbf{\hat L_z}
\end{align}

L'ultimo passaggio delle (3.42) discende dalle (3.15).

Siccome naturalmente $[L_x, L_x]=0$ potremo scrivere in modo compatto

\begin{equation}
[\mathbf{\hat L_i}, \mathbf{\hat L_j}]= i \hbar \epsilon_{i,j,k} \mathbf{\hat L_k}
\end{equation}

\section{Il modulo quadro del momento angolare}

Abbiamo visto in queste pagine che le varie componenti del momento angolare non sono osservabili compatibili tra loro, il che è un problema che limita le nostre capacità di studiare le rotazioni liberamente. Per risolverlo introduciamo l'operatore $\mathbf{L^2}$, definito come

\begin{equation}
\mathbf{L^2} = \mathbf{L_x}^2 + \mathbf{L_y}^2 + \mathbf{L_z}^2
\end{equation}

Perché è utile? Vediamo come si comporta con le componenti di $\mathbf{\hat L}$:

\begin{align}
[\mathbf{L_x}, \mathbf{L^2}] {}&= [\mathbf{L_x}, \mathbf{L_x}^2 + \mathbf{L_y}^2 + \mathbf{L_z}^2]= \nonumber \\
&= [\mathbf{L_x}, \mathbf{L_x}^2] + [\mathbf{L_x},\mathbf{L_y}^2] + [\mathbf{L_x},\mathbf{L_z}^2] 
\end{align}

Ricordando le proprietà dei commutatori avremo

\begin{align}
[\mathbf{L_x}, \mathbf{L_x}^2] {}&= [\mathbf{L_x}, \mathbf{L_x}]\mathbf{L_x} + \mathbf{L_x}[\mathbf{L_x}, \mathbf{L_x}] = 0 \\
[\mathbf{L_x}, \mathbf{L_y}^2] &= [\mathbf{L_x}, \mathbf{L_y}]\mathbf{L_y} + \mathbf{L_y}[\mathbf{L_x}, \mathbf{L_y}] = \nonumber \\
&= \mathbf{L_x}\mathbf{L_y^2} - \mathbf{L_y}\mathbf{L_x}\mathbf{L_y} + \mathbf{L_y}\mathbf{L_x}\mathbf{L_y} - \mathbf{L_x}\mathbf{L_y^2} = 0 \\
[\mathbf{L_x}, \mathbf{L_z}^2] &= 0 \:\: \text{per lo stesso motivo}
\end{align}

Lo stesso discorso vale anche per le altre componenti, possiamo quindi scrivere

\begin{equation}
[\mathbf{L_i}, \mathbf{L^2}] = 0
\end{equation}

Abbiamo quindi trovato un'operatore compatibile con tutte le componenti del momento angolare, il che ci permette di studiare contemporaneamente il momento angolare totale che una sua componente!

\newpage

\section{Autovalori ed autofunzioni del momento angolare}

\subsection{Ripasso sulle coordinate sferiche}

Per studiare problemi legati al momento angolare conviene mettersi in coordinate sferiche:


\begin{align}
\left\{\begin{array}{ccc}
x={}& r\sin(\theta)\cos(\phi) \\ 
y= &  r\sin(\theta)\sin(\phi) \\
z= &  r\cos(\theta) \qquad \;\:  \\
\end{array}\right. 
\end{align}

\begin{align}
\left\{\begin{array}{ccc}
dx={}& \sin(\theta)\cos(\phi)dr + r\cos(\theta)\cos(\phi) d\theta - r \sin(\theta)\sin(\phi) d\phi \\ 
dy= &  \sin(\theta)\sin(\phi)dr + r \cos(\theta)\sin(\phi)d\theta + r\sin(\theta)\cos(\phi) d \phi\\
dz= &  \cos(\theta) dr - r\sin(\theta) d\theta \qquad \qquad \qquad\qquad\qquad\qquad\qquad \\
\end{array}\right. 
\end{align}

Da cui ricaviamo

\begin{align}
\left\{\begin{array}{ccc}
dr={}& \sin(\theta)\cos(\phi)dx + \sin(\theta)\sin(\phi) dy + \cos(\theta)dz \qquad \\
d\theta =& r\cos(\theta)\cos(\phi)dx + r\cos(\theta)\sin(\phi)dy - r\sin(\theta) dz\\
d\phi =& -r \sin(\theta)\sin(\phi) dx +  r \sin(\theta)\cos(\phi) dy \qquad \qquad \quad \:
\end{array}\right. 
\end{align}

\begin{align}
\left\{\begin{array}{ccc}
\partial x ={}& \sin(\theta)\cos(\phi) \partial r + \frac{1}{2} \cos(\theta)\cos(\phi) \partial \theta - \frac{1}{2}\frac{\sin(\phi)}{\sin(\theta)}\partial \phi \\
\partial y=& \sin(\theta)\sin(\phi) \partial r + \frac{1}{2} \cos(\theta)\sin(\phi) \partial \theta - \frac{1}{2}\frac{\cos(\phi)}{r\sin(\theta)}\partial \phi \\
\partial z=& \cos(\theta)\partial r - \frac{1}{2} \sin (\theta) \partial \theta \qquad\qquad\qquad\qquad\qquad\quad
\end{array}\right. 
\end{align}


Utilizzando le coordinate sferiche possiamo riscrivere le componenti del momento angolare e il modulo quadro come:

\begin{align}
\left\{\begin{array}{ccc}
\mathbf{L_x}={}&  -i \hbar \left(
-\sin(\phi)\frac{\partial}{\partial \theta} - cotan(\theta)\cos(\phi)\frac{\partial}{\partial \phi}
\right) \\
\mathbf{L_y}=&  -i \hbar \left(
\cos(\phi)\frac{\partial}{\partial \theta} - cotan(\theta)\sin(\phi)\frac{\partial}{\partial \phi}
\right)\;\;\; \\
\mathbf{L_z}=&  -i \hbar \frac{\partial}{\partial \phi} \qquad\qquad\qquad\qquad\qquad\qquad\;\;\;\;\;\; \\
\mathbf{L^2}=& - \hbar^2 \left[\frac{1}{\sin(\theta)}\frac{\partial}{\partial \theta}\left(\sin(\theta)\frac{\partial}{\partial \theta}\right) + \frac{1}{\sin(\theta)^2}\frac{\partial^2}{\partial \phi^2} 
\right]
\end{array}\right. 
\end{align}

Notiamo subito come non ci sia dipendenza radiale, e come sia preferibile lavorare con la componente z per semplificare il carico di lavoro. Lavoreremo in RS per risolvere questo problema. \newpage


\subsection{Autovalori e autofunzioni di $L_z$}

Essendo $L_z$ funzione della sola $\phi$, ci aspettiamo che anche le sue autofunzioni lo saranno. Postuliamo che

\begin{equation}
L_z\psi_m(\phi)= m \hbar \psi_m(\phi)
\end{equation}
 
 Dal paragrafo precedente possiamo ricavare
 
 \begin{equation}
 L_z\psi_m(\phi)= -i \hbar \frac{\partial \psi_m (\phi)}{\partial \phi} = m \hbar \psi_m(\phi)
 \end{equation}
 
 da cui
 
 \begin{align}
{}& \frac{\partial \psi_m (\phi)}{\partial \phi} = i m \hbar \psi_m(\phi) \nonumber \\
&\downarrow \nonumber \\
&  \frac{\partial \psi_m (\phi)}{\psi_m(\phi)} = i m \hbar \partial \phi \nonumber \\
&\downarrow \nonumber \\
&\int_{\psi_m(0)}^{\psi_m(\phi)} \frac{\partial \psi_m (\phi)}{\psi_m(\phi)} = i m \hbar \int_{0}^{\phi} \partial \phi \nonumber \\
&\downarrow \nonumber \\
&\ln(\frac{\psi_m(\phi)}{\psi_m(0)}) = i m \phi \nonumber \\
&\downarrow \nonumber \\
&\frac{\psi_m(\phi)}{\psi_m(0)}= e^{i m \phi} \nonumber \\
&\downarrow \nonumber \\
&\psi_m(\phi)= \psi_m(0) e^{i m \phi}
\end{align}

Quanto varrà $\psi_m(0)$? Imponiamo la condizione di normalizzazione:

\begin{align}
|\psi_m(\phi)|^2= \int_{0}^{2\pi} d\phi \;|\psi_m(0)|^2 e^{-i m \phi}e^{i m \phi} = \int_{0}^{2\pi} d\phi \;|\psi_m(0)|^2 = 1
\end{align}

Da cui ricaviamo

\begin{equation}
|\psi_m(0)|^2 2\pi = 1 \rightarrow |\psi_m(0)|^2 = \frac{1}{2\pi}
\end{equation}

Possiamo quindi definire le autofunzioni di $\mathbf{L_z}$ come

\begin{equation}
\psi_m(\phi)= \frac{1}{\sqrt{2\pi}} e^{i m \phi}
\end{equation}

Siccome ci troviamo in uno spazio $L_2$, le $\psi_m(\phi)$ devono essere continue, questo vuol dire che

\begin{equation}
\psi_m(2\pi)= \psi_m(0) \rightarrow e^{2i m \pi}= 1
\end{equation}

E questo impone che 

\begin{equation}
m= 0, \pm 1, \pm 2, \dots , \pm n  \in Z
\end{equation}

\newpage

\subsection{Autovalori e autofunzioni di $L^2$}

Nel calcolo delle autofunzioni di $L^2$ il problema si complica, essendo presente la dipendenza da $\theta$. Iniziamo ricordando che $L^2$ e $L_z$ commutano, e quindi hanno un insieme di autofunzioni simultanee. Imponiamo quindi che

\begin{align}
\left\{\begin{array}{ccc}
L^2 Y_{l,m}(\theta, \phi)= \hbar^2 l(l+1) Y_{l,m} (\theta, \phi)\\
L_z Y_{l,m}(\theta, \phi)= \hbar m Y_{l,m}(\theta, \phi) \qquad \; \: \\
Y_{l,m}(\theta, \phi)= \Phi(\theta)\psi(\phi) \qquad \; \; \,
\end{array}\right. 
\end{align}

Iniziamo scrivendo

\begin{align}
{}&\hbar^2 \left[\frac{1}{\sin(\theta)}\frac{\partial}{\partial \theta}\left(\sin(\theta)\frac{\partial}{\partial \theta}\right) + \frac{1}{\sin^2(\theta)}\frac{\partial^2}{\partial \phi^2} 
\right] Y_{l,m}(\theta, \phi)= \hbar^2 l(l+1) Y_{l,m} (\theta, \phi) \nonumber \\
&\downarrow \nonumber \\
&\left[\frac{1}{\sin(\theta)}\frac{\partial}{\partial \theta}\left(\sin(\theta)\frac{\partial}{\partial \theta}\right) + \frac{1}{\sin^2(\theta)}\frac{\partial^2}{\partial \phi^2} 
\right] \Phi(\theta)\psi(\phi)= l(l+1) \Phi(\theta)\psi(\phi)
\end{align}

Già sappiamo che

\begin{align}
\left\{\begin{array}{ccc}
\psi_m(\phi)= \frac{1}{\sqrt{2\pi}} e^{i m \phi} \\
\psi'_m(\phi)= \frac{i m}{\sqrt{2\pi}} e^{i m \phi} \\
\psi''_m(\phi)= \frac{-m^2}{\sqrt{2\pi}} e^{i m \phi}
\end{array}\right. 
\end{align}

e quindi possiamo riscrivere la (3.65) come

\begin{align}
\left[
\frac{1}{\sin(\theta)}\frac{\partial}{\partial \theta}\left(\sin(\theta)\frac{\partial}{\partial \theta}\right)
 - \frac{m^2}{\sin^2(\theta)} + l(l+1) \right]\Phi(\theta)=0
\end{align}
 
Applicando il cambio di variabili

\begin{align}
\left\{\begin{array}{ccc}
\theta= arcos(\omega)\\
\Phi(\theta)= P_{l,m}(\omega)
\end{array}\right. 
\end{align}

Otteniamo, tramite conti che \textbf{non svolgeremo ancora}:

\begin{align}
\left[
(1 - \omega^2)\frac{\partial^2}{\partial \omega^2} - 2\omega \frac{\partial}{\partial \omega}+ l(l+1) - \frac{m^2}{1-\omega^2} \right]P_{l,m}(\omega)=0
\end{align}

Le cui soluzioni sono le cosidette \textbf{Equazioni differenziali di Legendre}:

\begin{align}
{}&P_{l,m}(\omega) = (1 - \omega^2)^{\frac{|m|}{2}} \left( \frac{\partial}{\partial \omega} \right)^{|m|}p_l(\omega) \\
&p_l(\omega)=\frac{1}{2^l \; l!} \left( \frac{\partial}{\partial \omega} \right)^2 (\omega^2 -1)^l
\end{align}

Possiamo quindi concludere scrivendo

\begin{equation}
Y_l^m(\theta, \phi)= \frac{P_{l,m}}{\sqrt{2\pi}}e^{im\phi}
\end{equation}

Le autofunzioni appartententi a questa famiglia prendono il nome di \textbf{armoniche sferiche}.

Un'ultima considerazione da fare è la seguente: 

\begin{align}
L^2 = L_x^2 + L_y^2 + L_z^2 \geq L_z^2 \rightarrow l(l+1) \geq m^2
\end{align}

\newpage  
  
\subsection{Operatori di innalzamento e abbassamento del momento angolare}

Abbiamo trovato come studiare la componente z del momento angolare. Ma come studiamo le altre due componenti, una volta fissati gli assi di riferimento? Introduciamo i seguenti operatori:

\begin{align}
\mathbf{L_{\pm}}=  \mathbf{L_x} \pm i\mathbf{L_y} \rightarrow L_\pm= (yp_z - zp_y) \pm i(zp_x - xp_z)
\end{align}

non sono hermitiani, ma solo aggiunti, infatti

\begin{equation}
\left\{
\begin{array}{ccc}
L^*_+ = L_-\\
L^*_-=L_+
\end{array} \right.
\end{equation}

Rispettano inoltre le seguenti regole di commutazione:

\begin{align}
[L_{\pm}, L^2] {}&= [L_x, L^2] \pm i[L_y, L^2] = 0  \\
\nonumber \\
[L_+,L_-] &= [ L_x + iL_y, L_x - iL_y]= \nonumber \\
&= [L_x, L_x] -i[L_x,L_y]+i[L_y,L_x]+[L_y,L_y]= \nonumber \\
&= -2i[L_y,L_x]= -2i(-i)\hbar L_z= 2 \hbar L_z \\
\nonumber \\
[L_z, L_\pm] &= [L_z, L_x \pm i L_y]= [L_z, L_x] \pm i [L_z,L_y]= \nonumber \\
&= i \hbar L_y \pm i (-i \hbar) L_x = \hbar(iL_y \mp L_x)= \nonumber \\
&= \pm(L_x \pm iL_y)=  \pm \hbar L_\pm
\end{align}

Si può anche dimostrare che

\begin{equation}
L_\pm L_\mp = L^2 - L_z^2 \pm \hbar L_z
\end{equation}

Notiamo ora che 


\begin{equation}
L^2 L_\pm Y_l^m(\theta, \phi)= L_\pm L^2 Y_l^m(\theta, \phi) = \hbar^2l(l+1) L_\pm Y_l^m(\theta, \phi)
\end{equation}


quindi possiamo dire che $L_\pm Y_l^m(\theta, \phi)$ è ancora autofunzione di $L^2$.

Invece se facciamo

\begin{equation}
L_z L_\pm Y_l^m(\theta, \phi)= (L_+ L_z \pm \hbar L_+)Y_l^m(\theta, \phi) = \hbar (m \pm 1) L_\pm Y_l^m(\theta, \phi) 
\end{equation}


Quindi, a parole:

\begin{enumerate}
	\item La mutua azione di $L^2$ e $L_\pm$ agisce sul sistema ma non varia l'autovalore del primo
	\item Invece se agiscono $L_z$ e $L_\pm$ l'autovalore di $L_z$ viene innalzato o abbassato di $\hbar$
\end{enumerate}

L'azione di $L_\pm$ (definiti quindi come \textbf{operatori di salita e discesa}) può essere descritta come:

\begin{equation}
\psi ' = L_\pm \psi = L_\pm Y_l^m(\theta, \phi) = C_{l,m}^\pm Y_l^{m \pm 1}(\theta, \phi)
\end{equation}

Ma quanto valgono le costanti $C_{l,m}^\pm$? si può procedere in due modi: 

\begin{enumerate}
	\item Scriviamo $L_\pm$ in coordinate polari:
	
	\begin{equation}
	L_\pm = \hbar e^{\pm i \phi} \left[
	\pm \frac{\partial}{\partial \theta} + i \frac{\cos(\theta)}{\sin(\theta)}\frac{\partial}{\partial \phi}
	\right]
	\end{equation}
	 e lo si applica alle armoniche sferiche, ottenendo alla fine il risultato
	 
	 \begin{equation}
	 L_\pm Y_l^m(\theta, \phi) = \hbar \sqrt{l(l+1) - m(m \pm 1)} Y_l^{m \pm 1}
	 \end{equation}
	
	\newpage
	
	\item Passare in notazione di Dirac, e notare che
	
	\begin{align}
	\left\{
	\begin{array}{ccc}
	\mathbf{\hat L_+} \ket{l,m} = C_{l,m}^+ \ket{l,m+1}\\
	\mathbf{\hat L_-} \ket{l,m} = C_{l,m}^- \ket{l,m-1}
	\end{array}
	\right.
		\end{align}
	
	con
	\begin{align}
		\left\{
	\begin{array}{ccc}
	\bra{l,m} \mathbf{\hat L_+^\dagger}= \bra{l,m} \mathbf{\hat L_-} =\bra{l,m-1} (C_{l,m}^-)^*\\
	\bra{l,m} \mathbf{\hat L_-^\dagger}= \bra{l,m} \mathbf{\hat L_+} =\bra{l,m+1} (C_{l,m}^+)^*
	\end{array}
	\right.
	\end{align}
	
	combinandole otteniamo
	\begin{align}
	\bra{l,m} \mathbf{\hat L_-} \mathbf{\hat L_+} \ket{l,m} {}&= |C_{l,m}^+|^2 \braket{l, m+1 | l, m+1} = \nonumber \\
	&= \bra{l,m} \mathbf{\hat L_-}\mathbf{\hat L_+} \ket{l,m}= \nonumber \\
	&= \bra{l,m} \mathbf{\hat L^2} - \mathbf{\hat L_z^2} -\hbar \mathbf{\hat L_z}  \ket{l,m}= \nonumber \\
	&= \hbar^2 l(l+1) -  \hbar^2 m^2 -  \hbar^2 m \nonumber \\
	&\downarrow \nonumber \\
	|C_{l,m}^+|^2 &= \hbar^2 l(l+1) -  \hbar^2 m^2 -  \hbar^2 m \nonumber \\
	&\downarrow \nonumber \\
	C_{l,m}^+ &= \hbar \sqrt{l(l+1) - m(m + 1)} 
	\end{align}
	
	con discorso analogo per $C_{l,m}^-$
\end{enumerate}